\documentclass[a4paper,twoside,12pt]{article}
\usepackage{../z/zed-cm}
\usepackage{graphicx}
\usepackage[nottoc,numbib]{tocbibind}
%\usepackage{hyperref}
\markboth{Draft}{Version 0.1}
\pagestyle{myheadings}
\begin{document}
\parskip 10 pt
\parindent 0 pt

\def\Slash{\slash\hspace{0pt}}

\title{Streaming Functions in riff}

\author{Glyn Normington}

\maketitle
% The following three commands ensure the title page is without a page number but page numbering starts here.
% Page numbers appear on subsequent pages, and are roman until the main body, which starts again at arabic 1.
\thispagestyle{empty}
\pagenumbering{roman}
\setcounter{page}{1}

%=============================================================================

This document provides a formal model of streaming functions in the riff FaaS project (\texttt{https://projectriff.io}).

% Alt-Cmd-M -- \emph{}
% Alt-Cmd-Z -- \zed{}
% Alt-Cmd-X -- \axdef{}
% Alt-Cmd-S -- \schema{}
% Alt-Shift-Cmd-T -- \texttt{}

% Type checking hacks
\newcommand{\true}{true}
\newcommand{\false}{false}
\renewcommand{\emptyset}{\varnothing}
%=============================================================================

\clearpage
\tableofcontents

\cleardoublepage
\pagenumbering{arabic}
\setcounter{page}{1}

%=============================================================================
\section{Introduction}

This document provides a formal model of streaming functions in the riff project.

%=============================================================================
\section{Overview of this document}

Event streaming is fraught with ambiguous terminology. For example, the word stream has multiple, overlapping
connotations. This document attempts to model streaming functions precisely. The model is abstract in the sense
that it does not capture all facets of streaming functions, but those it does capture, it captures precisely and
unambiguously.

The Z specification language is used to capture the model, but sufficient English text is also provided that readers who do not know Z should be able to understand the model. The appendix contains a summary of the Z notation.
For more information about Z, please consult the Z Manual (\texttt{https://www.cs.umd.edu/~mvz/handouts/z-manual.pdf}).
The model was type checked using \texttt{fuzz} (\texttt{https://bitbucket.org/Spivey/fuzz}).

%=============================================================================
\section{Fundamentals}

Streams consist of messages, but for our current purposes, the details of a message are irrelevant.
Similarly, errors are modelled but their details are not.
\begin{zed}
[ Message, Error ] 
\end{zed}

%=============================================================================
\newpage
\section{Reactive Streams}

A reactive stream consist of an ordered series of zero or more messages optionally followed by a completion signal or an error.
\begin{zed}
RStream ::= Next \ldata Message \cross RStream \rdata \mid \\
	 \t3 Complete \mid \\
	 \t3 Failed \ldata Error \rdata
\end{zed}

%=============================================================================
\section{Streaming Functions}

A streaming function takes a reactive stream as input and produces another reactive stream as output.
\begin{zed}
StreamingFun == RStream \fun RStream
\end{zed}

Streaming functions may be chained together to produce another streaming function.
The input stream of the chain is input to the 
first function in the chain. The output stream of the first function is the input
stream of the second function and so forth. The output stream of the chain is the output stream of the last function in the chain.
\begin{axdef}
Chain : \seq_1 StreamingFun \fun StreamingFun
\where
\forall f : StreamingFun; fs : \seq_1 StreamingFun @ \\
    \t1 Chain(\langle f \rangle) = f \land \\
    \t1 Chain(\langle f \rangle \cat fs) = Chain(fs) \circ f
\end{axdef}

%%%%%%%%%%%%%%%%%%%%%%%%%%%%%%%%%%%%%%%%%%%%%%%
%   A P P E N D I C E S
%%%%%%%%%%%%%%%%%%%%%%%%%%%%%%%%%%%%%%%%%%%%%%%

\clearpage

\appendix

%=============================================================================
%   Z   N O T A T I O N
%=============================================================================
\input{../z/z-notation.tex}

%=============================================================================
%   B I B L I O G R A P H Y
%=============================================================================
%\newpage
%\begin{flushleft}
%\begin{thebibliography}{99}
%\label{sec:references}
% `99' is a picture of the generated numeric references -- they are two digits in this bibliography
% If we had a hundred or more we would have used 999, or whatever.

%%  Example bibliography entry:
%\bibitem{knuth76}                                                        % citation callout, e.g.: \cite{knuth76}
%  Donald E. Knuth,                                                        % author
%  \emph{The computer as Master Mind}.                    % title
%  J. Recreational Mathematics, Vol.~9(1), 1976-1977. % publisher, or journal, volume and date

%\end{thebibliography}
%\end{flushleft}
\end{document}
