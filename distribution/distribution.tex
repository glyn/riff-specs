\documentclass[a4paper,twoside,12pt]{article}
\usepackage{../z/zed-cm}
\usepackage{graphicx}
\usepackage[nottoc,numbib]{tocbibind}
%\usepackage{hyperref}
\markboth{Draft}{Version 0.1}
\pagestyle{myheadings}
\begin{document}
\parskip 10 pt
\parindent 0 pt

\def\Slash{\slash\hspace{0pt}}

\title{Image Registries}

\author{Glyn Normington}

\maketitle
% The following three commands ensure the title page is without a page number but page numbering starts here.
% Page numbers appear on subsequent pages, and are roman until the main body, which starts again at arabic 1.
\thispagestyle{empty}
\pagenumbering{roman}
\setcounter{page}{1}

%=============================================================================

This document provides a formal model of image registries.

% Alt-Cmd-M -- \emph{}
% Alt-Cmd-Z -- \zed{}
% Alt-Cmd-X -- \axdef{}
% Alt-Cmd-S -- \schema{}
% Alt-Shift-Cmd-T -- \texttt{}

% Type checking hacks
\newcommand{\true}{true}
\newcommand{\false}{false}
\renewcommand{\emptyset}{\varnothing}
%=============================================================================

\clearpage
\tableofcontents

\cleardoublepage
\pagenumbering{arabic}
\setcounter{page}{1}

%=============================================================================
\section{Introduction}

This document provides a formal model of image registries.

%=============================================================================
\section{Overview of this document}

Docker Inc.\ introduced container images and registries to hold them and these were later standardised as part of the Open Container Initiative.

This document models image references, repositories, and registries. It covers digests and tags.

The Z specification language is used to capture the model, but sufficient English text is also provided that readers who do not know Z should be able to understand the model. The appendix contains a summary of the Z notation.
For more information about Z, please consult the Z Manual (\texttt{https://www.cs.umd.edu/~mvz/handouts/z-manual.pdf}).
The model was type checked using \texttt{fuzz} (\texttt{https://bitbucket.org/Spivey/fuzz}).

%=============================================================================
\section{Fundamentals}

Images are opaque blobs as far as we are concerned here - the decomposition into layers is ignored.
Similarly, cryptographic hashes, or \textit{hexes} to use the terminology of the OCI Distribution specification, tags, and (registry) hostnames and paths are modelled, but their details are not.
\begin{zed}
[ Image, Hex, Tag, Hostname, Path ] 
\end{zed}

There is a special reserved tag.
\begin{axdef}
    Latest : Tag
\end{axdef}

%=============================================================================
\newpage
\section{Content Digests}

A content digest is a combination of a cryptographic hash function (such as \texttt{SHA-256}) or "algorithm, and the hash output by such a function. 
\begin{schema}{ContentDigest}
    alg: Image \fun Hex \\
    hash: Hex \\ 
\end{schema}

The idea is that a content digest $d$ identifies an image $i$ if and only if:\\
%%unchecked
\begin{zed}
\t1 d.alg~i = d.hash
\end{zed}

An optional content digest is modelled as a datatype.
\begin{zed}
    OptionalContentDigest ::= None | Dig \ldata ContentDigest \rdata
\end{zed}

%=============================================================================
\newpage
\section{Repositories}

A repository is a collection of images indexed by content digest and by tag.
\begin{schema}{Repo}
    cd: ContentDigest \pfun Image \\
    tag: Tag \pfun Image \\
\where
    \forall d : \dom cd @ d.alg~(cd~d) = d.hash \\
    \ran tag \subseteq \ran cd \\
\end{schema}
The content digests identify the corresponding images. Each image identified by a tag is also identified by a content digest.

Initially, a repository is empty.
\begin{schema}{RepoInit}
  Repo' \\
\where
  cd' = \emptyset \\
  tag' = \emptyset \\
\end{schema}
 
 An image is added to a repository by \textit{pushing} it.
\begin{schema}{RepoPush}
  \Delta Repo \\
  i? : Image \\
  d? : ContentDigest \\
  t? : Tag \\
\where
  cd' = cd \oplus \{ d? \mapsto i? \} \\
  tag' = tag \oplus \{ t? \mapsto i? \} \\
\end{schema}

An image is retrieved from a repository by \textit{pulling} it.

We can either pull using a content digest
\begin{schema}{RepoPullByDigest}
  \Xi Repo \\
  d? : OptionalContentDigest \\
  i! : Image \\
\where
  \exists d : ContentDigest | d? = Dig~d  \land d \in \dom cd @ i! = cd~d \\
\end{schema}

or, if a content digest is not supplied, by using a tag.
\begin{schema}{RepoPullByTag}
  \Xi Repo \\
  t? : Tag \\
  d? : OptionalContentDigest \\
  i! : Image \\
\where
  d? = None \\
  t? \in \dom tag \implies i! = tag~t? \\
\end{schema}

\begin{zed}
  RepoPullOk \defs RepoPullByDigest \lor RepoPullByTag
\end{zed}

%=============================================================================
\newpage
\section{Registries}

A registry is a collection of repositories index by path.
\begin{schema}{Registry}
    repo: Path \fun Repo \\
\end{schema}
Paths which do not exist are modelled as pointing to empty repositories.

Initially a registry has only empty repositories.
\begin{schema}{RegistryInit}
  Registry' \\
\where
  \forall p : Path @ \\
  \t1 \exists RepoInit @ \\
  \t2 repo'~p = \theta Repo' \\
\end{schema}

We define a promotion schema which operates on a single repository in a registry.
\begin{schema}{RegistryPromote}
  \Delta Registry \\
  \Delta Repo \\
  p? : Path \\
\where
  p? \in \dom repo \\
  \theta Repo = repo~p? \\
  repo' = repo \oplus \{ p? \mapsto \theta Repo' \} \\
\end{schema}

We then promote the push and pull operations.
\begin{zed}
  RegistryPush \defs \exists \Delta Repo @ RepoPush \land RegistryPromote \\
  RegistryPullOk \defs \exists \Delta Repo @ RepoPullOk \land RegistryPromote \\
\end{zed}
  
Registries are arranged in a network indexed by hostname.
\begin{schema}{Net}
    reg: Hostname \pfun Registry \\
\end{schema}

Initiallly, there are no registries in the network.
\begin{schema}{NetInit}
  Net'
\where
  reg' = \emptyset
\end{schema}

We can add an empty registry to the network.
\begin{schema}{NetAddRegistryOk}
  \Delta Net \\
  h? : Hostname \\
\where
  h? \notin \dom reg \\
  \exists RegistryInit @ reg' = reg \cup \{ h? \mapsto \theta Registry' \} \\
\end{schema}

We can also remove a registry from the network.
\begin{schema}{NetRemoveRegistryOk}
  \Delta Net \\
  h? : Hostname \\
\where
  h? \in \dom reg \\
  reg' = \{ h? \} \ndres reg \\
\end{schema}

We define a promotion schema which operates on a single registry in a network.
\begin{schema}{NetPromote}
  \Delta Net \\
  \Delta Registry \\
  h? : Hostname \\
\where
  h? \in \dom reg \\
  \theta Registry = reg~h? \\
  reg' = reg \oplus \{ h? \mapsto \theta Registry' \} \\
\end{schema}

Finally, we promote the push and pull operations to work on a network.
\begin{zed}
  NetPushOk \defs \exists \Delta Registry @ RegistryPush \land NetPromote \\
  NetPullOk \defs \exists \Delta Registry @ RegistryPullOk \land NetPromote
\end{zed}
Pushing can fail if there is no registry with the input hostname.

%=============================================================================
\newpage
\section{Image References}

An image reference identifies an image in a registry.
\begin{schema}{Ref}
    host: Hostname \\
    path: Path \\
    tag: Tag \\
    dig : OptionalContentDigest \\
\end{schema}
A tag is always logically present, but if it is omitted from the textual representation of an image reference, it defaults to $Latest$.
A content digest may be part of an image reference or may be omitted.

So far the push and pull operations have accumulated several input parameters.
\begin{schema}{PushParms}
    h? : Hostname \\
    p? : Path \\
    t? : Tag \\
    d? : ContentDigest \\
\end{schema}
\begin{schema}{PullParms}
    h? : Hostname \\
    p? : Path \\
    t? : Tag \\
    d? : OptionalContentDigest \\
\end{schema}

An image reference is mapped to push input parameters as follows.
\begin{schema}{RefPushParms}
    r? : Ref \\
    PushParms \\
\where
    h? = r?.host \\
    p? = r?.path \\
    t? = r?.tag \\
    r?.dig = None \\
\end{schema}
Pushing is not allowed with a content digest in the image reference.

An image reference is mapped to pull input parameters as follows.
\begin{schema}{RefPullParms}
    r? : Ref \\
    PullParms \\
\where
    h? = r?.host \\
    p? = r?.path \\
    t? = r?.tag \\
    d? = r?.dig \\
\end{schema}

Push can then be reframed to take an image reference.
\begin{zed}
  RefPushOk \defs \exists PushParms @ NetPushOk \land RefPushParms \\
  RefPullOk \defs \exists PullParms @ NetPullOk \land RefPullParms \\
\end{zed}


%%%%%%%%%%%%%%%%%%%%%%%%%%%%%%%%%%%%%%%%%%%%%%%
%   A P P E N D I C E S
%%%%%%%%%%%%%%%%%%%%%%%%%%%%%%%%%%%%%%%%%%%%%%%

\clearpage

\appendix

%=============================================================================
%   Z   N O T A T I O N
%=============================================================================
\section{Z Notation}
\label{sec:znot}
{\scriptsize
\makeatletter % the following code is taken from Mike Spivey's zed.tex

\def\symtab{\setbox0=\vbox\bgroup \def\\{\cr}
        \halign\bgroup\strut$##$\hfil&\quad##\hfil\cr}
\def\endsymtab{\crcr\egroup\egroup
        \dimen0=\ht0 \divide\dimen0 by2 \advance\dimen0 by\ht\strutbox
        \splittopskip=\ht\strutbox \vbadness=10000
        \predisplaypenalty=0
        $$\halign{##\cr\hbox to\linewidth{%
                \valign{##\vfil\cr
                        \setbox1=\vsplit0 to\dimen0 \unvbox1\cr
                        \noalign{\hfil}\unvbox0\cr
                        \noalign{\hfil}}}\cr
                \noalign{\prevdepth=\dp\strutbox}}$$
        \global\@ignoretrue}

\makeatother

Numbers:
\begin{symtab}
        \nat & \verb/Natural numbers/ \{\verb/0,1,.../\} \\
%       \num & \verb/Integers (...,-1,0,1,...)/ \\
%       \nat_1 & \verb/Positive natural numbers/ \\
%       \upto & \verb/integral range/ \\
%       + & \verb/Addition/\quad\hfill 3 \\
%       - & \verb/Subtraction/\quad\hfill 3 \\
%       * & \verb/Multiply/\quad\hfill 4 \\
%       \div & \verb/Remainder/\quad\hfill 4 \\
%       \mod & \verb/Modulus/\quad\hfill 4 \\
%       < & \verb/Less than/ \\
%       > & \verb/Greater than/ \\
%       \leq & \verb/Less than or equal/ \\
%       \geq & \verb/Greater than or equal/ \\
%       \neq & \verb/Inequality/ \\
\end{symtab}
Propositional logic and the schema calculus:
\begin{symtab}
%       \lnot & \verb/Not/ \\
        \ldots\land\ldots & \verb/And/ \\
        \ldots\lor\ldots & \verb/Or/ \\
        \ldots\implies\ldots & \verb/Implies/ \\
%       \iff & \verb/If and only if/ \\
        \forall..\mid..\spot.. & \verb/For all/ \\
        \exists..\mid..\spot.. & \verb/There exists/ \\
%       \exists_1..\mid..\spot.. & \verb/There exists unique/ \\
        \ldots\hide\ldots & \verb/Hiding/ \\
%       \project & \verb/\project/ \\
%       \pre & \verb/\pre/ \\
%       \semi & \verb/\semi/
        \ldots\defs\ldots & \verb/Schema definition/ \\
        \ldots==\ldots & \verb/Abbreviation/ \\
        \ldots::=\ldots\mid\ldots & \verb/Free type definition/ \\
        \ldata\ldots\rdata & \verb/Free type injection/ \\
        [\ldots] & \verb/Given sets/ \\
        ',?,!,_0\ldots_9 & \verb/Schema decorations/ \\
        \ldots\shows\ldots & \verb/theorem/ \\
        \theta\ldots & \verb/Binding formation/ \\
        \lambda\ldots & \verb/Function definition/ \\
        \mu\ldots & \verb/Mu-expression/ \\
        \Delta\ldots & \verb/State change/ \\
        \Xi\ldots & \verb/Invariant state change/ \\
\end{symtab}
Sets and sequences:
%and bags:
\begin{symtab}
        \{\ldots\} & \verb/Set/ \\
        \{..\mid..\spot..\} & \verb/Set comprehension/ \\
        \power\ldots & \verb/Set of subsets of/ \\
%       \power_1 & \verb/Non-empty subsets of/ \\
%       \finset & \verb/Finite sets/ \\
%       \finset_1 & \verb/Non-empty finite sets/ \\
        \emptyset & \verb/Empty set/ \\
        \ldots\cross\ldots & \verb/Cartesian product/ \\
        \ldots\in\ldots & \verb/Set membership/ \\
        \ldots\notin\ldots & \verb/Set non-membership/ \\
        \ldots\cup\ldots & \verb/Union/ \\
        \ldots\cap\ldots & \verb/Intersection/ \\
        \ldots\setminus\ldots & \verb/Set difference/ \\
        \bigcup\ldots & \verb/Distributed union/ \\
%       \bigcap & \verb/Distributed intersection/ \\
        \#\ldots & \verb/Cardinality/ \\
%       \dcat & \verb/Distributed sequence concatenation/
        \ldots\subseteq\ldots & \verb/Subset/ \\
        \ldots\subset\ldots & \verb/Proper subset/ \\
        \ldots\partition\ldots & \verb/Set partition/ \\
        \seq & \verb/Sequences/ \\
%       \seq_1 & \verb/Non-empty sequences/ \\
%       \iseq & \verb/Injective sequences/ \\
        \langle\ldots\rangle & \verb/Sequence/ \\
%       \cat & \verb/Sequence concatenation/ \\
        \disjoint\ldots & \verb/Disjoint sequence of sets/ \\
%       \bag & \verb/Bags/ \\
%       \lbag\ldots\rbag & \verb/Bag/ \\
%       \inbag & \verb/Bag membership/ \\
\end{symtab}
%Here are the infix function symbols. Each symbol is
%shown with its priority:
%\begin{symtab}
%       \uplus & \verb/\uplus/ \\
%       \filter & \verb/Schema projection/ \\
%       \uminus & \verb/\uminus/
%\end{symtab}
Functions and relations:
\begin{symtab}
        \ldots\rel\ldots\quad\quad~~~  & \verb/Relation/ \\
        \ldots\pfun\ldots & \verb/Partial function/ \\
        \ldots\fun\ldots  & \verb/Total function/ \\
        \ldots\pinj\ldots & \verb/Partial injection/ \\
        \ldots\inj\ldots  & \verb/Injection/ \\
%       \psurj & \verb/Partial surjection/ \\
%       \surj & \verb/Surjection/ \\
%       \bij  & \verb/Bijection/ \\
%       \ffun & \verb/Finite partial function/ \\
%       \finj & \verb/Finite partial injection/ \\
        \dom\ldots & \verb/Domain/ \\
        \ran\ldots & \verb/Range/ \\
        \ldots\mapsto\ldots & \verb/maplet/ \\
        \ldots\inv & \verb/Relational inverse/ \\
%       \ldots\plus & \verb/Transitive closure/ \\
        \ldots\star & \verb/Reflexive-transitive/ \\
        ~           & \verb/closure/ \\
%       \ldots\bsup n \esup & \verb/Relational iteration/ \\
        \ldots\limg\ldots\rimg & \verb/Relational image/ \\
%       \comp & \verb/Forward relational composition/ \\
%       \circ & \verb/Relational composition/ \\
        \ldots\oplus\ldots & \verb/Functional overriding/ \\
        \ldots\dres\ldots & \verb/Domain restriction/ \\
        \ldots\rres\ldots & \verb/Range restriction/ \\
        \ldots\ndres\ldots & \verb/Domain subtraction/ \\
        \ldots\nrres\ldots & \verb/Range subtraction/ \\
%       \id & \verb/Identity relation/ \\
\end{symtab}
Axiomatic descriptions:
%%unchecked
\begin{axdef}
  Declarations
\where
  Predicates
\end{axdef}
Schema definitions:
%%unchecked
\begin{schema}{SchemaName}
  Declaration
\where
  Predicates
\end{schema}
}
\newpage

%=============================================================================
%   B I B L I O G R A P H Y
%=============================================================================
%\newpage
%\begin{flushleft}
%\begin{thebibliography}{99}
%\label{sec:references}
% `99' is a picture of the generated numeric references -- they are two digits in this bibliography
% If we had a hundred or more we would have used 999, or whatever.

%%  Example bibliography entry:
%\bibitem{knuth76}                                                        % citation callout, e.g.: \cite{knuth76}
%  Donald E. Knuth,                                                        % author
%  \emph{The computer as Master Mind}.                    % title
%  J. Recreational Mathematics, Vol.~9(1), 1976-1977. % publisher, or journal, volume and date

%\end{thebibliography}
%\end{flushleft}
\end{document}
