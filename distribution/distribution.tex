\documentclass[a4paper,twoside,12pt]{article}
\usepackage{../z/zed-cm}
\usepackage{graphicx}
\usepackage[nottoc,numbib]{tocbibind}
%\usepackage{hyperref}
\markboth{Version 0.3}{Version 0.3}
\pagestyle{myheadings}
\begin{document}
\parskip 10 pt
\parindent 0 pt

\def\Slash{\slash\hspace{0pt}}

\title{Image Registries}

\author{Glyn Normington}

\maketitle
% The following three commands ensure the title page is without a page number but page numbering starts here.
% Page numbers appear on subsequent pages, and are roman until the main body, which starts again at arabic 1.
\thispagestyle{empty}
\pagenumbering{roman}
\setcounter{page}{1}

%=============================================================================

This document provides a formal model of some aspects of \textit{Docker} image registries.

I am indebted to Chris Frost and Steve Powell for their helpful feedback.

% Alt-Cmd-M -- \emph{}
% Alt-Cmd-Z -- \zed{}
% Alt-Cmd-X -- \axdef{}
% Alt-Cmd-S -- \schema{}
% Alt-Shift-Cmd-T -- \texttt{}

% Type checking hacks
\newcommand{\true}{true}
\newcommand{\false}{false}
\renewcommand{\emptyset}{\varnothing}
%=============================================================================

\clearpage
\tableofcontents

\cleardoublepage
\pagenumbering{arabic}
\setcounter{page}{1}

%=============================================================================
\section{Introduction}

This document provides a formal model of image registries.

Docker Inc. \cite{docker}\footnote{See the bibliography in Appendix \ref{sec:references}.} introduced \textit{container images} and \textit{registries} to hold them and these were later standardised as part of the Open Container Initiative \cite{oci}. The reader is assumed to have a basic understanding of how images are used.

This document models image references, repositories, and registries. It covers digests and tags.

The Z specification language is used to describe the model, but sufficient English text is also provided that readers who do not know Z should be able to understand what's going on. Appendix \ref{sec:znot} contains a summary of the Z notation.
For more information about Z, please consult the Z Manual \cite{zman}.
The model was type checked using \texttt{fuzz} \cite{fuzz}.

%=============================================================================
\section{Fundamentals}

Images are opaque blobs as far as we are concerned here - the decomposition into layers is ignored.
Similarly, cryptographic hashes (or \textit{hexes}, to use the terminology of the OCI Distribution specification), tags, and (registry) hostnames and paths are modelled in terms of their usage, but not their implementation.
\begin{zed}
[ Image, Hex, Tag, Hostname, Path ] 
\end{zed}

There is a special reserved tag.
\begin{axdef}
    Latest : Tag
\end{axdef}

%=============================================================================
\newpage
\section{Content Digests}

A content digest is a combination of a cryptographic hash function (such as \texttt{SHA-256}) or ``algorithm'', and the hash output by such a function. 
\begin{schema}{ContentDigest}
    alg: Image \fun Hex \\
    hash: Hex \\ 
\end{schema}
The algorithm in a content digest is represented as a string which names one of a limited set of algorithms, but we ignore that here
and pretend the algorithm is part of the content digest.

The idea is that a content digest $d$ identifies an image $i$ if and only if applying the hash function (from the content digest) to the image produces the hash output in the content digest:\\
%%unchecked
\begin{zed}
\t1 d.alg~i = d.hash
\end{zed}

The limited set of algorithms is chosen so that given an image $i$ with a content digest $d$, it is extremely difficult, if not impossible, to construct
a distinct image $j$ with the same content digest:\\
%%unchecked
\begin{zed}
\t1 i \neq j \land d.alg~i = d.hash = d.alg~j
\end{zed}

An optional content digest is modelled as a datatype.
\begin{zed}
    OptionalContentDigest ::= None | Dig \ldata ContentDigest \rdata \\
\end{zed}

We add a convenient function for extracting the content digest from an optional content digest which is ``present''.
\begin{zed}
    Cd == Dig\inv \\
\end{zed}

%=============================================================================
\newpage
\section{Repositories}

A repository is a collection of images indexed by content digest and by tag.
\begin{schema}{Repo}
    cd: ContentDigest \pinj Image \\
    tag: Tag \pfun Image \\
\where
    \forall d : \dom cd @ d.alg~(cd~d) = d.hash \\
    \ran tag \subseteq \ran cd \\
\end{schema}
The content digests identify the corresponding images and there is only one content digest per image (since this is chosen when the image added
to the repository, as we shall soon see). Each image identified by a tag is also identified by a content digest.

Initially, a repository is empty.
\begin{schema}{RepoInit}
  Repo' \\
\where
  cd' = \emptyset \\
  tag' = \emptyset \\
\end{schema}
 
An image is added to a repository by \textit{pushing} it. The operation takes any image and any tag and returns a content digest of the image.
\begin{schema}{RepoPush}
  \Delta Repo \\
  i? : Image \\
  t? : Tag \\
  d! : ContentDigest
\where
  tag' = tag \oplus \{ t? \mapsto i? \} \\
  cd' = cd \oplus \{ d! \mapsto i? \} \\
\end{schema}
The tag may be omitted, in which case it defaults to $Latest$, although this is not modelled. Note that the invariant of $Repo'$ ensures that the output digest identifies the input image. There is, in theory, some non-determinism in the choice of algorithm, but for OCI compliant implementations, \texttt{SHA-256} is used everywhere.

\newpage
An image is retrieved from a repository by \textit{pulling} it.

We can pull an image using a content digest.
\begin{schema}{RepoPullByDigest}
  \Xi Repo \\
  d? : OptionalContentDigest \\
  i! : Image \\
\where
  d? \in \dom Cd \\
  i! = cd(Cd~d?) \\
\end{schema}
The content digest is provided and is used to identify the output image.

Alternatively, we can pull and image using a tag.
\begin{schema}{RepoPullByTag}
  \Xi Repo \\
  t? : Tag \\
  d? : OptionalContentDigest \\
  i! : Image \\
\where
  d? = None \\
  i! = tag~t? \\
\end{schema}
The content digest is not provided. The tag is used to identify the output image.

A successful pull operation uses either a content digest or a tag.
\begin{zed}
  RepoPullOk \defs RepoPullByDigest \lor RepoPullByTag
\end{zed}
If a content digest is provided, the image is pulled by digest and the tag is ignored. Failure cases, such as ``not found'', are omitted from the model.

%=============================================================================
\newpage
\section{Registries}

A registry is a collection of repositories indexed by path.
\begin{schema}{Registry}
    repo: Path \pfun Repo \\
\end{schema}

An empty registry has no repositories.
\begin{zed}
  EmptyRegistry \defs [ Registry | repo = \emptyset ]
\end{zed}

Initially a registry has no repositories.
\begin{zed}
  RegistryInit \defs EmptyRegistry'
\end{zed}

An empty repository may be added at a given path.
\begin{schema}{RegistryAddRepoOk}
  \Delta Registry \\
  p? : Path \\
\where
  p? \notin \dom repo \\
  \exists RepoInit @ repo' = repo \cup \{ p? \mapsto \theta Repo' \}
\end{schema}
Some registry implementations, such as Amazon's ECR, require a repository to be created before an image can be pushed to the repository.
Other registry implementations such as Docker's registry and Google's GCR will create a repository automatically when the first image is pushed to it. We do not model automatic repository creation.

\pagebreak
We define a promotion schema\footnote{Promotion schemas are used to turn operations on a particular type into operations on a collection of the type which operate on a single member of the collection and leave the rest unchanged.} which operates on a single repository in a registry.
\begin{schema}{RegistryPromote}
  \Delta Registry \\
  \Delta Repo \\
  p? : Path \\
\where
  p? \in \dom repo \\
  \theta Repo = repo~p? \\
  repo' = repo \oplus \{ p? \mapsto \theta Repo' \} \\
\end{schema}
The path must be valid and the registry is preserved except for the repository identified by the path which may be updated.

We then promote the repository push and pull operations to operate on a registry.
\begin{zed}
  RegistryPush \defs \exists \Delta Repo @ RepoPush \land RegistryPromote \\
  RegistryPullOk \defs \exists \Delta Repo @ RepoPullOk \land RegistryPromote \\
\end{zed}
  
Registries are accessed remotely in a network via their hostnames.
\begin{schema}{Net}
    reg: Hostname \pfun Registry \\
\end{schema}

Initiallly, there are no registries in the network.
\begin{schema}{NetInit}
  Net'
\where
  reg' = \emptyset
\end{schema}

We can add an empty registry to the network.
\begin{schema}{NetAddRegistryOk}
  \Delta Net \\
  h? : Hostname \\
\where
  h? \notin \dom reg \\
  \exists EmptyRegistry @ reg' = reg \cup \{ h? \mapsto \theta EmptyRegistry \} \\
\end{schema}

We can also remove a registry from the network.
\begin{schema}{NetRemoveRegistryOk}
  \Delta Net \\
  h? : Hostname \\
\where
  h? \in \dom reg \\
  reg' = \{ h? \} \ndres reg \\
\end{schema}

We define a promotion schema which operates on a single registry in a network.
\begin{schema}{NetPromote}
  \Delta Net \\
  \Delta Registry \\
  h? : Hostname \\
\where
  h? \in \dom reg \\
  \theta Registry = reg~h? \\
  reg' = reg \oplus \{ h? \mapsto \theta Registry' \} \\
\end{schema}
The hostname must be valid and the network is preserved except for the registry identified by the hostname which may be updated.

Finally, we promote the push and pull operations to work on a network.
\begin{zed}
  NetPushOk \defs \exists \Delta Registry @ RegistryPush \land NetPromote \\
  NetPullOk \defs \exists \Delta Registry @ RegistryPullOk \land NetPromote
\end{zed}
Pushing and pulling fail if there is no registry with the input hostname.

%=============================================================================
\newpage
\section{Image References}

An image reference identifies an image in a registry. Let's remind ourselves what image references look like.

An image reference consists of a hostname (with optional port) and a path. The image reference may also contain a tag and/or a digest. The hostname determines the network location of a registry. The path consists of one or more components separated by forward slashes. The first component is sometimes, by convention for certain registries, a user name providing access control to the image.

Let's look at some examples:
\begin{itemize}
\item The image name \texttt{docker.io/istio/proxyv2} refers to an image with user name \texttt{istio} residing in the docker hub registry at \texttt{docker.io}.

\item The image name \texttt{projectriff/builder:v1} is short-hand for\linebreak
\texttt{docker.io/projectriff/builder:v1} which refers to an image with user name \texttt{projectriff} also residing at \texttt{docker.io}. The image has tag \texttt{v1}.

\item The image name \texttt{gcr.io/cf-elafros/knative-releases/github.com}\linebreak\texttt{/knative/serving/cmd/autoscaler@sha256:deadbeefdeadbeef}\linebreak\texttt{deadbeefdeadbeefdeadbeefdeadbeefdeadbeefdeadbeef} refers to an image with user name \texttt{cf-elafros} residing at \texttt{gcr.io}. The image has a (made up) \texttt{SHA-256} digest.
\end{itemize}

For the purposes of our model, an image reference consists of a hostname, a path, a tag, and an optional content digest.
\begin{schema}{Ref}
    host: Hostname \\
    path: Path \\
    tag: Tag \\
    dig : OptionalContentDigest \\
\end{schema}
A tag is always logically present, but if it is omitted from the textual representation of an image reference, it defaults to $Latest$.
A content digest may be part of an image reference or may be omitted.

So far the push and pull operations have accumulated several input parameters.
\begin{schema}{PushParms}
    h? : Hostname \\
    p? : Path \\
    t? : Tag \\
\end{schema}
\begin{schema}{PullParms}
    h? : Hostname \\
    p? : Path \\
    t? : Tag \\
    d? : OptionalContentDigest \\
\end{schema}

An image reference is related to push input parameters as follows.
\begin{schema}{RefPushParms}
    r? : Ref \\
    PushParms \\
\where
    h? = r?.host \\
    p? = r?.path \\
    t? = r?.tag \\
    r?.dig = None \\
\end{schema}
Pushing is not allowed if the image reference has a content digest.

An image reference is related to pull input parameters as follows.
\begin{schema}{RefPullParms}
    r? : Ref \\
    PullParms \\
\where
    h? = r?.host \\
    p? = r?.path \\
    t? = r?.tag \\
    d? = r?.dig \\
\end{schema}

Push and pull can then be reframed to take an image reference instead of the corresponding input parameters.
\begin{zed}
  RefPushOk \defs \exists PushParms @ NetPushOk \land RefPushParms \\
  RefPullOk \defs \exists PullParms @ NetPullOk \land RefPullParms \\
\end{zed}


%%%%%%%%%%%%%%%%%%%%%%%%%%%%%%%%%%%%%%%%%%%%%%%
%   A P P E N D I C E S
%%%%%%%%%%%%%%%%%%%%%%%%%%%%%%%%%%%%%%%%%%%%%%%

\clearpage

\appendix

%=============================================================================
%   B I B L I O G R A P H Y
%=============================================================================
\newpage
\begin{flushleft}
\renewcommand\refname{Bibliography}
\begin{thebibliography}{99}
\label{sec:references}
% `99' is a picture of the generated numeric references -- they are two digits in this bibliography
% If we had a hundred or more we would have used 999, or whatever.

%%  Example bibliography entry:
%\bibitem{knuth76}                                                        % citation callout, e.g.: \cite{knuth76}
%  Donald E. Knuth,                                                        % author
%  \emph{The computer as Master Mind}.                    % title
%  J. Recreational Mathematics, Vol.~9(1), 1976-1977. % publisher, or journal, volume and date

\bibitem{docker}
  Various authors,
  \emph{Docker Inc.},
  \texttt{https://www.docker.com/}.

\bibitem{oci}
  Various authors,
  \emph{Open Container Initiative},
  \texttt{https://www.opencontainers.org/}.

\bibitem{zman}
  Mike Spivey,
  \emph{The Z Manual},
  \texttt{https://www.cs.umd.edu/ mvz/handouts/z-manual.pdf}.

\bibitem{fuzz}
  Mike Spivey,
  \emph{The fuzz type-checker for Z},
  \texttt{https://bitbucket.org/Spivey/fuzz}.


\end{thebibliography}
\end{flushleft}

%=============================================================================
%   Z   N O T A T I O N
%=============================================================================
\newpage
\section{Z Notation}
\label{sec:znot}
{\scriptsize
\makeatletter % the following code is taken from Mike Spivey's zed.tex

\def\symtab{\setbox0=\vbox\bgroup \def\\{\cr}
        \halign\bgroup\strut$##$\hfil&\quad##\hfil\cr}
\def\endsymtab{\crcr\egroup\egroup
        \dimen0=\ht0 \divide\dimen0 by2 \advance\dimen0 by\ht\strutbox
        \splittopskip=\ht\strutbox \vbadness=10000
        \predisplaypenalty=0
        $$\halign{##\cr\hbox to\linewidth{%
                \valign{##\vfil\cr
                        \setbox1=\vsplit0 to\dimen0 \unvbox1\cr
                        \noalign{\hfil}\unvbox0\cr
                        \noalign{\hfil}}}\cr
                \noalign{\prevdepth=\dp\strutbox}}$$
        \global\@ignoretrue}

\makeatother

Numbers:
\begin{symtab}
        \nat & \verb/Natural numbers/ \{\verb/0,1,.../\} \\
%       \num & \verb/Integers (...,-1,0,1,...)/ \\
%       \nat_1 & \verb/Positive natural numbers/ \\
%       \upto & \verb/integral range/ \\
%       + & \verb/Addition/\quad\hfill 3 \\
%       - & \verb/Subtraction/\quad\hfill 3 \\
%       * & \verb/Multiply/\quad\hfill 4 \\
%       \div & \verb/Remainder/\quad\hfill 4 \\
%       \mod & \verb/Modulus/\quad\hfill 4 \\
%       < & \verb/Less than/ \\
%       > & \verb/Greater than/ \\
%       \leq & \verb/Less than or equal/ \\
%       \geq & \verb/Greater than or equal/ \\
%       \neq & \verb/Inequality/ \\
\end{symtab}
Propositional logic and the schema calculus:
\begin{symtab}
%       \lnot & \verb/Not/ \\
        \ldots\land\ldots & \verb/And/ \\
        \ldots\lor\ldots & \verb/Or/ \\
        \ldots\implies\ldots & \verb/Implies/ \\
%       \iff & \verb/If and only if/ \\
        \forall..\mid..\spot.. & \verb/For all/ \\
        \exists..\mid..\spot.. & \verb/There exists/ \\
%       \exists_1..\mid..\spot.. & \verb/There exists unique/ \\
        \ldots\hide\ldots & \verb/Hiding/ \\
%       \project & \verb/\project/ \\
%       \pre & \verb/\pre/ \\
%       \semi & \verb/\semi/
        \ldots\defs\ldots & \verb/Schema definition/ \\
        \ldots==\ldots & \verb/Abbreviation/ \\
        \ldots::=\ldots\mid\ldots & \verb/Free type definition/ \\
        \ldata\ldots\rdata & \verb/Free type injection/ \\
        [\ldots] & \verb/Given sets/ \\
        ',?,!,_0\ldots_9 & \verb/Schema decorations/ \\
        \ldots\shows\ldots & \verb/theorem/ \\
        \theta\ldots & \verb/Binding formation/ \\
        \lambda\ldots & \verb/Function definition/ \\
        \mu\ldots & \verb/Mu-expression/ \\
        \Delta\ldots & \verb/State change/ \\
        \Xi\ldots & \verb/Invariant state change/ \\
\end{symtab}
Sets and sequences:
%and bags:
\begin{symtab}
        \{\ldots\} & \verb/Set/ \\
        \{..\mid..\spot..\} & \verb/Set comprehension/ \\
        \power\ldots & \verb/Set of subsets of/ \\
%       \power_1 & \verb/Non-empty subsets of/ \\
%       \finset & \verb/Finite sets/ \\
%       \finset_1 & \verb/Non-empty finite sets/ \\
        \emptyset & \verb/Empty set/ \\
        \ldots\cross\ldots & \verb/Cartesian product/ \\
        \ldots\in\ldots & \verb/Set membership/ \\
        \ldots\notin\ldots & \verb/Set non-membership/ \\
        \ldots\cup\ldots & \verb/Union/ \\
        \ldots\cap\ldots & \verb/Intersection/ \\
        \ldots\setminus\ldots & \verb/Set difference/ \\
        \bigcup\ldots & \verb/Distributed union/ \\
%       \bigcap & \verb/Distributed intersection/ \\
        \#\ldots & \verb/Cardinality/ \\
%       \dcat & \verb/Distributed sequence concatenation/
        \ldots\subseteq\ldots & \verb/Subset/ \\
        \ldots\subset\ldots & \verb/Proper subset/ \\
        \ldots\partition\ldots & \verb/Set partition/ \\
        \seq & \verb/Sequences/ \\
%       \seq_1 & \verb/Non-empty sequences/ \\
%       \iseq & \verb/Injective sequences/ \\
        \langle\ldots\rangle & \verb/Sequence/ \\
%       \cat & \verb/Sequence concatenation/ \\
        \disjoint\ldots & \verb/Disjoint sequence of sets/ \\
%       \bag & \verb/Bags/ \\
%       \lbag\ldots\rbag & \verb/Bag/ \\
%       \inbag & \verb/Bag membership/ \\
\end{symtab}
%Here are the infix function symbols. Each symbol is
%shown with its priority:
%\begin{symtab}
%       \uplus & \verb/\uplus/ \\
%       \filter & \verb/Schema projection/ \\
%       \uminus & \verb/\uminus/
%\end{symtab}
Functions and relations:
\begin{symtab}
        \ldots\rel\ldots\quad\quad~~~  & \verb/Relation/ \\
        \ldots\pfun\ldots & \verb/Partial function/ \\
        \ldots\fun\ldots  & \verb/Total function/ \\
        \ldots\pinj\ldots & \verb/Partial injection/ \\
        \ldots\inj\ldots  & \verb/Injection/ \\
%       \psurj & \verb/Partial surjection/ \\
%       \surj & \verb/Surjection/ \\
%       \bij  & \verb/Bijection/ \\
%       \ffun & \verb/Finite partial function/ \\
%       \finj & \verb/Finite partial injection/ \\
        \dom\ldots & \verb/Domain/ \\
        \ran\ldots & \verb/Range/ \\
        \ldots\mapsto\ldots & \verb/maplet/ \\
        \ldots\inv & \verb/Relational inverse/ \\
%       \ldots\plus & \verb/Transitive closure/ \\
        \ldots\star & \verb/Reflexive-transitive/ \\
        ~           & \verb/closure/ \\
%       \ldots\bsup n \esup & \verb/Relational iteration/ \\
        \ldots\limg\ldots\rimg & \verb/Relational image/ \\
%       \comp & \verb/Forward relational composition/ \\
%       \circ & \verb/Relational composition/ \\
        \ldots\oplus\ldots & \verb/Functional overriding/ \\
        \ldots\dres\ldots & \verb/Domain restriction/ \\
        \ldots\rres\ldots & \verb/Range restriction/ \\
        \ldots\ndres\ldots & \verb/Domain subtraction/ \\
        \ldots\nrres\ldots & \verb/Range subtraction/ \\
%       \id & \verb/Identity relation/ \\
\end{symtab}
Axiomatic descriptions:
%%unchecked
\begin{axdef}
  Declarations
\where
  Predicates
\end{axdef}
Schema definitions:
%%unchecked
\begin{schema}{SchemaName}
  Declaration
\where
  Predicates
\end{schema}
}
\newpage

\end{document}
